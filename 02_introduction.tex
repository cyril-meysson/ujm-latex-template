\chapter{Introduction}

\summary{Un court résumé du chapitre peut se rédiger ici.}

\section{Insérer des références bibliographiques}
Ceci est le début d'un paragraphe qui inclut la citation d'une \enquote{expression propre à un auteur particulier\autocite{bauberotLaiciteCommePacte2001}}. La même référence\autocite{bauberotLaiciteCommePacte2001} citée à nouveau est référencée par \frquote{\emph{Ibid.}}. La citation suivante, plus longue, est signalée par un retrait plus important :
\begin{quotation}
Nouvelle tuile pour la SNCM. Hier, la justice européenne a estimé que l'aide de 76 millions d'euros accordée par l'Etat pour la restructuration de la Société nationale maritime Corse-Méditerranée en 2002 n'est pas légale. Motif : en déclarant, en juillet 2003, cette recapitalisation compatible avec le Marché commun, la Commission européenne a omis de prendre en compte 12 millions de recettes récupérées par la compagnie publique, grâce à la cession d'actifs immobiliers. L'aide n'a donc pas été limitée au minimum, comme l'exigent les règles européennes. C'est pourquoi le tribunal de première instance de la Cour européenne a annulé l'autorisation de la Commission.\autocite{henrySNCMRattrapeePar2005}
\end{quotation}
Le paragraphe continue ensuite sans espacement de première ligne. Citer une nouvelle fois le premier auteur\autocite{bauberotLaiciteCommePacte2001} ajoute \frquote{\textit{op. cit.}} à la référence. Une autre citation plus détaillée issue d'une revue\autocite{walterParcMonsieurZola1978} ou bien d'un livre.\autocite{nicolasBretagneDestinEuropeen2001}

\section{Insérer des images}
Des images réalisées avec le logiciel \textit{SoX} au format \textit{.png} sont insérées ci-dessous, avec les figures \ref{fig-lin_sweep}, \ref{fig-lin_sweep-clip} et \ref{fig-lin_sweep-clip_OS} qui présentent l'effet du suréchantillonnage sur le repliement du spectre (ou \emph{aliasing}) lors de l'écrêtage d'un balayage sinusoïdal linéaire (ou \textit{linear sine sweep}) :

\begin{figure}[htbp]
  \centering
  \includegraphics[width=0.6 \textwidth]{images/linear_sweep.png}
  \caption{Balayage sinusoïdal linéaire.}
  \label{fig-lin_sweep}
\end{figure}

\begin{figure}[htbp]
  \centering
  \includegraphics[width=0.6 \textwidth]{images/linear_sweep-hard_clip.png}
  \caption{Balayage sinusoïdal linéaire écrêté de 3\textit{dBFS} sans 
  suréchantillonnage.}
  \label{fig-lin_sweep-clip}
\end{figure}

\begin{figure}[htbp]
  \centering
  \includegraphics[width=0.6 \textwidth]{images/linear_sweep-hard_clip_OS.png}
  \caption{Balayage sinusoïdal linéaire écrêté de 3\textit{dBFS} avec 
  suréchantillonnage de facteur 8.}
  \label{fig-lin_sweep-clip_OS}
\end{figure}
