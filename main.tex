\documentclass[12pt,a4paper,french]{book}

%%%%%%%%%%%%%%%%%%
% Packages LaTeX %
%%%%%%%%%%%%%%%%%%

% Dimensions et marges des pages :
\usepackage[a4paper,
margin=2cm,
bindingoffset=1cm,
includeheadfoot]{geometry}

% Polices et encodage :
\usepackage[utf8]{inputenc}
\usepackage[T1]{fontenc}
\usepackage{fourier}
\usepackage[babel=true]{microtype}
\usepackage{newunicodechar}
\newunicodechar{∙}{\textperiodcentered} % Support du point médian

% Mise en page :
\usepackage{setspace} % interlignes perso
\usepackage[defaultlines=4,all]{nowidow}
\usepackage{fancyhdr}
\usepackage{emptypage}
\usepackage{lipsum} % pour tester la mise en page

% Gestion des couleurs :
\usepackage[dvipsnames,usenames]{color}
% \usepackage[enable]{darkmode} % À désactiver avant l'export

% Inclusion d'images/figures :
\usepackage{float}
\usepackage{graphicx}
\pdfminorversion=7 % Permettre PDF > v1.5
\pdfimageresolution=300 % Images en 300 DPI
\usepackage{epsfig} % Support des fichiers vectoriels .eps
\usepackage[section]{placeins} % Ne pas séparer les éléments flottants de leur section

% Inclusion de tableaux :
\usepackage{booktabs}
\usepackage{tabularx}

% Normes typographiques :
\usepackage[backend=biber,
style=verbose-trad2,
singletitle=true,
isbn=false,
doi=false,
ibidpage=true,
autopunct=false,
]{biblatex}
\usepackage[autostyle=true]{csquotes} % Gestion des guillemets
\usepackage[french,noconfigs]{babel}
\addto\captionsfrench{\def\tablename{Tableau}} % Remplace "table" par "tableau"
\usepackage{mfirstuc} % Majuscules automatiques
\usepackage{etoolbox}

% Inclusion d'hyperliens et des métadonnées PDF :
\usepackage[unicode,
pdfauthor={Prénom NOM},
pdftitle={Titre du mémoire},
pdfsubject={Sujet du mémoire},
pdfkeywords={mots-clés}]{hyperref}

%%%%%%%%%%%%%%%%%%%%%%%%%%%%
% Détails de mise en forme %
%%%%%%%%%%%%%%%%%%%%%%%%%%%%

% Espacements sur page de garde :
\newcommand{\HRule}{\rule{\linewidth}{0.5mm}}

% Interligne 1.5 :
\onehalfspacing

% Citations plus petites
\AtBeginEnvironment{quotation}{\small}

% Profondeur de numérotation :
\setcounter{secnumdepth}{4} % dans le texte
\setcounter{tocdepth}{4}    % dans le sommaire

% En-têtes et pieds de page :
\fancyhf{}
\fancyhead[LE]{\itshape\nouppercase{\leftmark}}
\fancyhead[RO]{\itshape\nouppercase{\rightmark}}
\fancyhead[LO]{}
\fancyhead[RE]{}
\fancyfoot[LE,RO]{\thepage}
\renewcommand{\headrulewidth}{0.4pt}
\renewcommand{\footrulewidth}{0pt}
\setlength{\headheight}{15pt}

\fancypagestyle{plain}{% Pour corriger les pages en style plain
  \fancyhead{}\renewcommand{\headrule}{}%
  \renewcommand{\headrulewidth}{0.4pt}%
}

% Résumés de chapitres :
\newcommand{\summary}[1]{%
  \hfill%
  \begin{minipage}{0.8\linewidth}
    \emph{Résumé du chapitre : }#1%
  \end{minipage}%
  \par
  \newpage
}

% Noms des listes des tables et figures :
% \addto\captionsfrench{\renewcommand{\listfigurename}{Liste des figures}}
% \addto\captionsfrench{\renewcommand{\listtablename}{Liste des tables}}

%%%%%%%%%%%%%%%%%%%%%%%%%%%
% Normes bibliographiques %
%%%%%%%%%%%%%%%%%%%%%%%%%%%

\renewcommand*{\newunitpunct}{\addcomma\space} % utiliser la virgule comme délimiteur
\renewcommand*{\mkibid}{\emph} % Ibid. en italique
\renewcommand*{\revsdnamepunct}{} % Supprime les virgules entre noms et prénoms
\renewcommand*{\intitlepunct}{} % Supprime les deux points après in

% Ne conserve "in" que pour les articles issus d'ouvrages collectifs
\renewbibmacro*{in:}{%
  \ifentrytype{article}
    {}
    {
      \newunitpunct
      \printtext{\bibstring{in}\space}%
    }%
}

\renewbibmacro*{cite:title}{% Supprime le titre pour op. cit.
  \printtext[bibhyperlink]{%
    \ifsingletitle{}{\printfield[citetitle]{labeltitle}}%
    \setunit{\nametitledelim}%
    \bibstring[\mkibid]{opcit}}
}

% Ajoute date de publication originale au titre
\renewbibmacro*{title}{%
  \ifboolexpr{
    test {\iffieldundef{title}}
    and
    test {\iffieldundef{subtitle}}
  }
    {}
    {\printtext[title]{%
       \printfield[titlecase]{title}%
       \setunit{\subtitlepunct}%
       \printfield[titlecase]{subtitle}}%
       \iffieldundef{origyear}
    {}
    {%
       \setunit{\addspace}
       \printtext[parens]{\printfield{origyear}}}%
     \newunit}%
  \printfield{titleaddon}}

% Ajoute un séparateur entre le titre périodique et la date
\renewbibmacro*{issue+date}{%
  \printfield{issue}%
  \newunit
  \usebibmacro{date}%
  \newunit}

% Redéfinition des citations issues de collections 
\renewbibmacro*{byeditor+others}{%
  \ifnameundef{editor}
    {}
    {%\setunit{\addspace}%
     \printnames[labelname]{editor}% Initiales pour les prénoms des éditeurs
     \clearname{editor}%
     \setunit{\addspace}%
     \usebibmacro{byeditor+othersstrg}%
     \newunit}%
  \usebibmacro{byeditorx}%
  \usebibmacro{bytranslator+others}}

\DeclareBibliographyDriver{incollection}{%
  \usebibmacro{bibindex}%
  \usebibmacro{begentry}%
  \usebibmacro{author/translator+others}%
  \setunit{\labelnamepunct}\newblock
  \usebibmacro{title}%
  \newunit
  \printlist{language}%
  \newunit\newblock
  \usebibmacro{byauthor}%
  \newunit\newblock
  \usebibmacro{in:}%
  \usebibmacro{byeditor+others}%
  \newunit\newblock
  \usebibmacro{maintitle+booktitle}%
  \newunit\newblock
  \printfield{edition}%
  \newunit
  \iffieldundef{maintitle}
    {\printfield{volume}%
     \printfield{part}}
    {}%
  \newunit
  \printfield{volumes}%
  \newunit\newblock
  \printfield{note}%
  \newunit\newblock
  \printlist{location}%
  \iflistundef{publisher}
    {\setunit*{\addcomma\space}}
    {\setunit*{\addcolon\space}}%
  \printlist{publisher}%
  \setunit*{\addcomma\space}%
  \usebibmacro{series+number}% Nom de collection/série avant la date
  \newunit\newblock
  \usebibmacro{date}%
  \newunit
  \newunit\newblock
  \iftoggle{bbx:isbn}
    {\printfield{isbn}}
    {}%
  \newunit\newblock
  \usebibmacro{doi+eprint+url}%
  \newunit\newblock
  \usebibmacro{addendum+pubstate}%
  \setunit{\bibpagerefpunct}\newblock
  \usebibmacro{chapter+pages}% Numéros de pages à la fin
  \newunit\newblock
  \usebibmacro{pageref}%
  \newunit\newblock
  \iftoggle{bbx:related}
    {\usebibmacro{related:init}%
     \usebibmacro{related}}
    {}%
  \usebibmacro{finentry}}

% Remplace éd. par dir.
\DefineBibliographyStrings{french}{%
  byeditor           = {\mkbibparens{dir\adddot}}
}

\DeclareNameFormat{labelname}{% Ajoute les initiales du prénom pour ibid. et op. cit.
  \usebibmacro{name:family-given}
    {\namepartfamily}
    {\namepartgiveni}
    {\namepartprefix}
    {\namepartsuffixi}%
  \usebibmacro{name:andothers}
}

\DeclareNameAlias{sortname}{family-given} % Inverse les noms et prénoms
\DeclareNameAlias{default}{family-given} % idem

\DeclareFieldFormat[article,periodical,online]{title}{\enquote{#1}} % Guillemets pour les titres autres qu'ouvrages
\DeclareFieldFormat*{number}{\newunitpunct n° {#1}} % n°
\DeclareFieldFormat{url}{\textless\url{#1}\textgreater} % URLs entre crochets
\DeclareFieldFormat{urldate}{\newunitpunct consulté le {#1}} % Consulté le :
\DeclareListFormat[online]{organization}{\textit{{#1}}\space\lbrack en ligne\rbrack} % [en ligne] 
\DeclareFieldFormat{series}{coll.\addspace\enquote{\capitalisewords{#1}}\addcomma} % coll.
\DeclareFieldFormat{editortype}{\mkbibparens{#1}} % Parenthèses pour les éditeurs
\DeclareDelimFormat{editortypedelim}{\addspace} % Espace comme séparateur pour les rôles d'éditeurs
\DeclareListFormat{location}{#1\newunitpunct} % Délimiteur correct

% Importer la bibliographie
\addbibresource{biblio.bib}

%%%%%%%%%%%%%%%%%%%%%
% Début du document %
%%%%%%%%%%%%%%%%%%%%%

\begin{document}

%%%%%%%%%%%%%%%%%%%%%%%%
% Section introductive %
%%%%%%%%%%%%%%%%%%%%%%%%

\frontmatter 
\newgeometry{hmarginratio=1:1,margin=2cm} % marges symétriques en page de garde
\begin{titlepage}
  \begin{center}

    % Logo(s) :
    \includegraphics[width=0.8\textwidth]{./images/Fac.ALL-LOGO-CMJN}
    \hfill
    
    \textsc{\Large Mémoire de Master 1 Création Contemporaine et Nouvelles Technologies}\\[0.5cm]
    \textsc{\large Parcours Réalisateur en Informatique Musicale}\\[0.5cm]

    % Titre :
    \HRule \\[0.4cm]
    
    {\huge \bfseries Intitulé du mémoire\\[0.6cm]}
    
    \HRule \\[1.5cm]

    % Nom de l'étudiant et des encadrants :
    {\Large \textbf{Prénom \textsc{Nom}}}\\[1cm]
    \large
    \emph{Sous la direction de:}\\
    Prénom \textsc{Nom}, professeur des universités
    \vfill
    \small
    Université Jean Monnet -- Faculté Arts, Lettres, Langues

    Département de Musicologie

    2025--2026

  \end{center}
\end{titlepage}
\restoregeometry
\chapter*{Remerciements} % Ne pas inclure dans la table des matières

Les remerciements peuvent être rédigés ici, sur un ou plusieurs paragraphes.
\tableofcontents

%%%%%%%%%%%%%%%%%%%%%%
% Section principale %
%%%%%%%%%%%%%%%%%%%%%%

\mainmatter
\pagestyle{fancy} % Inclure un fichier par chapitre
\chapter{Introduction}

\summary{Un court résumé du chapitre peut se rédiger ici.}

\section{Insérer des références bibliographiques}
Ceci est le début d'un paragraphe qui inclut la citation d'une \enquote{expression propre à un auteur particulier\autocite{bauberotLaiciteCommePacte2001}}. La même référence\autocite{bauberotLaiciteCommePacte2001} citée à nouveau est référencée par \frquote{\emph{Ibid.}}. La citation suivante, plus longue, est signalée par un retrait plus important :
\begin{quotation}
Nouvelle tuile pour la SNCM. Hier, la justice européenne a estimé que l'aide de 76 millions d'euros accordée par l'Etat pour la restructuration de la Société nationale maritime Corse-Méditerranée en 2002 n'est pas légale. Motif : en déclarant, en juillet 2003, cette recapitalisation compatible avec le Marché commun, la Commission européenne a omis de prendre en compte 12 millions de recettes récupérées par la compagnie publique, grâce à la cession d'actifs immobiliers. L'aide n'a donc pas été limitée au minimum, comme l'exigent les règles européennes. C'est pourquoi le tribunal de première instance de la Cour européenne a annulé l'autorisation de la Commission.\autocite{henrySNCMRattrapeePar2005}
\end{quotation}
Le paragraphe continue ensuite sans espacement de première ligne. Citer une nouvelle fois le premier auteur\autocite{bauberotLaiciteCommePacte2001} ajoute \frquote{\textit{op. cit.}} à la référence. Une autre citation plus détaillée issue d'une revue\autocite{walterParcMonsieurZola1978} ou bien d'un livre.\autocite{nicolasBretagneDestinEuropeen2001}

\section{Insérer des images}
Des images réalisées avec le logiciel \textit{SoX} au format \textit{.png} sont insérées ci-dessous, avec les figures \ref{fig-lin_sweep}, \ref{fig-lin_sweep-clip} et \ref{fig-lin_sweep-clip_OS} qui présentent l'effet du suréchantillonnage sur le repliement du spectre (ou \emph{aliasing}) lors de l'écrêtage d'un balayage sinusoïdal linéaire (ou \textit{linear sine sweep}) :

\begin{figure}[htbp]
  \centering
  \includegraphics[width=0.6 \textwidth]{images/linear_sweep.png}
  \caption{Balayage sinusoïdal linéaire.}
  \label{fig-lin_sweep}
\end{figure}

\begin{figure}[htbp]
  \centering
  \includegraphics[width=0.6 \textwidth]{images/linear_sweep-hard_clip.png}
  \caption{Balayage sinusoïdal linéaire écrêté de 3\textit{dBFS} sans 
  suréchantillonnage.}
  \label{fig-lin_sweep-clip}
\end{figure}

\begin{figure}[htbp]
  \centering
  \includegraphics[width=0.6 \textwidth]{images/linear_sweep-hard_clip_OS.png}
  \caption{Balayage sinusoïdal linéaire écrêté de 3\textit{dBFS} avec 
  suréchantillonnage de facteur 8.}
  \label{fig-lin_sweep-clip_OS}
\end{figure}


%%%%%%%%%%%%%%%%%%
% Section finale %
%%%%%%%%%%%%%%%%%%

\backmatter

% Utiliser "type=" pour créer des sous-catégories bibliographiques
\printbibheading[heading=bibintoc]
\printbibliography[nottype=online, heading=subbibintoc, title=Ouvrages]
\printbibliography[type=online, heading=subbibintoc, title=Sitographie]

%%%%%%%%%%%%%%%%%%%
% Fin du document %
%%%%%%%%%%%%%%%%%%%

\end{document}